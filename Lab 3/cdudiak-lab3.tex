\documentclass{article}
\usepackage[pdftex,
        colorlinks=true,
        urlcolor=rltblue,       % \href{...}{...} external (URL)
        filecolor=rltgreen,     % \href{...} local file
        linkcolor=rltred,       % \ref{...} and \pageref{...}
        pdftitle={Phys 20 Lab 3},
        pdfauthor={Chris Dudiak},
        pdfproducer={pdfLaTeX},
        pagebackref,
        pdfpagemode=None,
        bookmarksopen=true]{hyperref}
\usepackage{color}
\usepackage{graphicx}
\usepackage{mathtools}
\usepackage{placeins}
\usepackage{listings}
\usepackage{verbatim}
\definecolor{rltred}{rgb}{0.75,0,0}
\definecolor{rltgreen}{rgb}{0,0.5,0}
\definecolor{rltblue}{rgb}{0,0,0.75}


\begin{document}

\title{Phys 20 Lab 3 - Numerical Solutions to ODEs: Spring Problem}
\author{Chris Dudiak}
\date{\today}
\maketitle

\section{Explicit Euler}
\begin{verbatim}
def springProp(z, h):
    # loop over the lists simultaneously and compute new values
    for i, (x, v) in enumerate(z):
        if (i + 1 >= N):
            break
        z[i + 1] = (x + h * v, v - h * x)
    return z
\end{verbatim}
This function sets the next values based on the old values. Using this function, the position and velocity are graphed vs time. For this set:
\begin{align*}
	x_0 &= 5.0\\
	v_0 &= 0.0\\
	h &= .01
\end{align*}
Unless otherwise stated. Plotting the propagation of the spring for 2000 steps, we get:

\begin{figure}[h!]
	\centering
	\includegraphics[scale=.6]{"explicit"}
	\caption{Explicit Evaluation of Euler's Method on Spring}
\end{figure} 

\FloatBarrier
\section{Analytical Solution}
For the analytic solution of $a = {d^2x \over dt^2} = -x$ with the initial conditions above and ${k\over m}  = 1$:
\begin{align*}
	&x(t) = 5.0\cos(t)\\
	&v(t) = -5.0\sin(t)\\
\end{align*}
Plotting the errors we find:
\begin{figure}[h!]
	\centering
	\includegraphics[scale=.6]{"errors"}
	\caption{Explicit Evaluation of Euler's Method Errors from Analytic Solution}
\end{figure} 
\FloatBarrier
As time increase, the maximum magnitude of the error in both position and velocity gets larger; this increase in maximum peaks is linear 
with time.

\section{Truncation Errors}
Using h = .0027, we see the following truncation errors:
\begin{figure}[h!]
	\centering
	\includegraphics[scale=.6]{"trunc"}
	\caption{Explicit Evaluation of Euler's Method Truncation Errors on h}
\end{figure} 
\FloatBarrier

As the value of h increases, the maximum error from the analytic solution increases as well. At extremely small h's, the error is almost 0 as far as the
floating point values can detect. The error grows quadratically as h increases though.

\section{Energy}
Looking at Total Energy $E = v^2 + x^2$ as a function of time:
\begin{figure}[h!]
	\centering
	\includegraphics[scale=.6]{"energy"}
	\caption{Explicit Evaluation of Euler's Method Energy}
\end{figure} 
\FloatBarrier
We see that the energy is increasing quadratically over time due to the squares in the position and velocities, which grow linearly. Eventually, this energy
would go to infinity rather than remain constant due to the errors in the floating point precision. At small values of h, the energy could remain near 25 
as expected, but the error would probably still propagate forward over time and build on itself.

\section{Implicit Euler}
Solving the implicit solution, we can find:
\begin{align*}
	\begin{pmatrix}
		x_{i + 1} \\
		v_{i + 1}
	\end{pmatrix}
	=
	\begin{pmatrix}
		1 & -h \\
		h & 1
	\end{pmatrix} ^{-1}
	\begin{pmatrix}
		x_{i} \\
		v_{i}
	\end{pmatrix}
\end{align*}
Using this, we can calculate the global errors and energy as before:
\begin{figure}[h!]
	\centering
	\includegraphics[scale=.5]{"positionComp"}
	\caption{Implicit and Explicit Euler's Method Position Errors}
\end{figure} 
\begin{figure}[h!]
	\centering
	\includegraphics[scale=.5]{"velocityComp"}
	\caption{Implicit and Explicit Euler's Method Velocity Errors}
\end{figure} 
\begin{figure}[h!]
	\centering
	\includegraphics[scale=.5]{"energyComp"}
	\caption{Implicit and Explicit Euler's Method Energy}
\end{figure} 
\FloatBarrier
As we can see, the implicit errors are of the same form but mirror the explicit errors given that they are calculated from new values rather
than old values. They evolve similarly although the peaks for the explicit errors are increasing faster as they are unbounded. The energies
however go in opposite directions; the implicit energy tends to zero rather than positive infinity for the same reason. The energy is expressed as 
$v^2 + x^2$ so the implicit energy decays to zero but the explicit blows up indefinitely to infinity.


\section{Conservation of Energy}\ 
\subsection{Implicit and Explicit Phase-Space}
Since the energy is not conserved in the implicit and explicit cases, they will not form complete circles in phase space. The explicit
form, which tends to infinite energy, spirals outward, and the implicit solution, which tends to zero spirals inward.

\begin{figure}[h!]
	\centering
	\includegraphics[scale=.5]{"ephase"}
	\caption{Explicit Phase $v_0 = 0.0, d_0 = 5.0$: Spirals Out}
\end{figure} 
\begin{figure}[h!]
	\centering
	\includegraphics[scale=.5]{"iphase"}
	\caption{Implicit Phase $v_0 = 0.0, d_0 = 5.0$: Spirals In}
\end{figure} 
\FloatBarrier

\subsection{Symplectic Euler}

The Symplectic Euler method uses:
\begin{align*}
	&x_{i+1} = x_i + h v_i\\
	&v_{i+1} = v_i - h x_{i+1}\\
\end{align*}

This is generated by the following code:
\begin{verbatim}
def springSymp(z, h):
    # loop over the lists simultaneously and compute new values
    for i, (x, v) in enumerate(z):
        if (i + 1 >= N):
            break
        x_next = x + h * v
        z[i + 1] = (x_next, v - h * x_next)
    return z
\end{verbatim}

Since energy is now conserved much better, this creates circles in phase space rather than spirals:
\begin{figure}[h!]
	\centering
	\includegraphics[scale=.5]{"sphase"}
	\caption{Symplectic Phase $v_0 = 0.0, d_0 = 5.0$, h = .5: Energy Conserved}
\end{figure} 
\FloatBarrier

We see that the two curves agree at the initial conditions but that the symplectic phase is slightly askew from the exact result. At points it
has lower energy and at points higher energy. If we varied the initial conditions, we would expect both to still agree at those points but the oscillations
could vary in magnitude based on the initial conditions and expected total energy.
 
\subsection{Symplectic Energy}
Looking at the energy evolution of the Symplectic Euler, we find: 
\begin{figure}[h!]
	\centering
	\includegraphics[scale=.5]{"symenergy"}
	\caption{Symplectic Energy $v_0 = 0.0, d_0 = 5.0$: Energy Oscillates}
\end{figure} 
\FloatBarrier

Rather than remaining perfectly constant, the energy oscillates sinusoidally. Unlike the implicit and explicit methods, it does not go
to infinity over time. The maximum energy deviation is about .05\% of the initial total energy, which is a small variation.

\section{Code and Info}
\subsection{Output}
The program reports the following:
\verbatiminput{output.txt}

\subsection{Code}
Code for this week's set:
\lstinputlisting[language=Python]{spring.py}

\end{document}
